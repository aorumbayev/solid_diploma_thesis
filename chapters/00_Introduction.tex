\chapter*{Introduction}
\label{chap:introduction}
% \addcontentsline{toc}{chapter}{Introduction}

The World Wide Web as we know it started on March 12, 1989, by Sir Tim Berners-Lee. What began as a proposal, eventually ended up being one of the most important technological achievements of a century. Nowadays, the ability to access information online is often an effortless process. People can easily use a search engine to look up articles, connect with anyone in a matter of seconds using social network platforms, consume gigabytes of media \cite{www_foundation_intro}. Furthermore, they can upload, store, and share any data online. When it comes to storing data online, an average Internet user will most probably rely on companies providing their storage solutions in the \textit{Cloud}. The majority of popular cloud storage providers like \textit{Google}, \textit{Dropbox} or \textit{Microsoft OneDrive} are centralized. Centralization is in no means a matter of a concern for an average internet user.

On the contrary, it usually provides a better user experience when the user has all of his relevant data stored in a centralized \textit{data silo}. The term \textit{data silo} often describes a fixed repository of data entirely under control of a single department while being isolated from the rest of the organization. On the other hand, it raises a lot of privacy concerns when it comes to explaining who owns the data stored under such cloud storage. With the growth of large software corporations and an increase in the dominance of their services and technologies where billions of users upload, store and share data under their centralized silos, a lot of examples of disadvantages of centralization were demonstrated for the past several years. For example, an infamous \textit{Facebook–Cambridge Analytica} data scandal in early 2018 \footnote{\url{https://en.wikipedia.org/wiki/Facebook–Cambridge_Analytica_data_scandal}}, demonstrated how millions of \textit{Facebook} profiles were analyzed without any consent and later targeted for political advertising. In a certain sense, the fact that data was under control of a single organization and stored in centralized fashion played an important role in raising privacy concerns and making people more cautious about relying on centralized providers to own their data.

On August 10, 2016, another ambitious project was launched by Sir Tim Berners-Lee called \solid{} \footnote{\url{https://solidproject.org}}. The goal of the project is to make World Wide Web decentralized, improve data ownership on the internet, and give the full control over the data back to the users by providing an alternative to dominating storage technologies relying on centralized data silos. Even though the idea might be appealed as too ambitious, over the years, the project has grown from several proof-of-concept proposals to a large community of developers actively contributing and expanding the project every day. The \solid{} project, by definition, represents multiple things at once. In formal terms, it is a set of specifications, principles, conventions, and tools for building \textit{decentralized social applications} while relying on principles of \textit{Linked Data}. The term \solid{} by itself stands for \textit{Social Linked Data}, where the term \textit{Linked Data} stands for yet another concept that will be heavily utilized in this thesis project. Linked Data is a method for publishing structured data in a way that preserves the semantics of the data. This semantic description is implemented by the use of \textit{vocabularies} \footnote{\url{https://www.w3.org/standards/semanticweb/ontology}}, which are usually specified by the \acrshort{W3C} as web standards. However, anyone can create and register their vocabulary, for example, in an open catalog like \textit{\gls{LOV}} \footnote{\url{https://lov.linkeddata.es/dataset/lov/}}. Linked data is usually dispersed across many sites on the internet. Each site usually contains only a part of the entire data available. Thus a machine or a person trying to interpret the data as a whole needs to link this incomplete information together using unique entity identifiers shared across the data stores. Another common term usually associated with Linked Data is \textit{Linked Open Data}, in contrast with Linked Data, it follows the \textit{five star Linked Open Data} model \footnote{\url{https://www.w3.org/community/webize/2014/01/17/what-is-5-star-linked-data/}}. It represents Linked Data that is publicly accessible to everyone.

In November 2018, a group of five Master students, including the myself, at Faculty of Mathematics and Physics at Charles Univesity in Prague, started a university software project called \textit{\gls{LPA}}. The project was focused around development of a web app that would simplify the interactions with Linked Data and other concepts of \textit{Semantic Web} \footnote{\url{https://www.w3.org/standards/semanticweb/}} for average \textit{lay} users and provide an intuitive and straightforward way to visualize Linked Data expressed in \textit{RDF} format for various needs. The initial goals of that project did not involve any plans to rely on any decentralized storages, such as ones represented by \solid{}. However, over time the set of functional and non-functional requirements related to dealing with massive amounts of Linked Data expressed in RDF format, ability to share the applications created with the platform while maintaining the control over them, and most importantly storing and sharing the \lpa{} platform data became specific enough to become a perfect fit to use \solid{} as a primary technology for storing the data. 
 
\section*{Goal of the thesis}
% Solid mentions
The main goal of the following thesis is to provide a decentralized web-based storage solution based on \solid{} project for the \lpa{} platform and demonstrate the full benefits of decentralized storage and the \solid{} technology. The results of the practical work must satisfy all defined functional and non-functional requirements stated by \lpa{} project. The complete implementation of the solution implementing the requirements must be provided and described in detail as a part of this thesis. It is important to note that the \solid{} project is still in the early stages of development, and its specifications are being updated regularly. Therefore, aside from fitting the needs of \lpa{} requirements, the implemented toolset must be generic enough for the possibility to be extended for usage in any application using \solid{}. Further mentions of the practical part of this thesis are going to be called as \textit{\gls{LPS}}.

\section*{Structure}

The structure consist of nine main chapters that can be described as follows:
\begin{enumerate}
    \item \textit{Preliminaries}, Basic introduction to core concepts of \textit{Semantic Web} and \solid{} that will be appearing throughout the rest of the thesis. The remaining sections of the chapter are dedicated to general description and main concepts of \lpa{}. That project defines the core requirements for this thesis, therefore it is important to make sure that all \lpa{} specific terms and definitions are explained before proceeding to further chapters.
    
    \item \textit{Overview of decentralized web-based storage technologies}, the chapter provides an overview of decentralized software technologies that are alternative or opposed to concepts of \solid{}. Additionally, the chapter also provides the core ideas on why \solid{} turned out to be a perfect fit for \lpa{}.

    \item \textit{Analysis}, chapter is an overview of all functional and non-functional requirements stated by \lpa{} as they are the main users of the \lpas{}.

    \item \textit{Architecture}, chapter continues the \textit{Analysis} chapter by describing the architecture of the solution. Additionally, the first section of the chapter provides a review of technologies, frameworks, and libraries that were under assumption during the design of the architecture and that later used in \textit{Implementation} chapter.  

    \item \textit{Implementation}, chapter continues the \textit{Architecture} chapter by diving into implementation details and how both specifications and architecture from the previous chapters were implemented into an actual project. 
    
    \item \textit{Evaluation}, chapter demonstrates the benefits of choosing \solid{} as a main technology for decentralized storage in \lpa{} project. The results, recognition, and notable achievements are also described in detail in this chapter.
    
    \item \textit{Testing}, chapter describes everything related to testing in \lpas{} starting from generic conventions and finalizing on specific implementation details.
    
    \item \textit{Documentation}, the chapter describes the various documentation resources written during the implementation of \lpa{} and provides references to access them.
     
     \item \textit{Future work} chapter provides an overview of how the project planned to be improved in future. 
\end{enumerate}

The summary of results achieved is provided in \textit{Conclusion} chapter, briefly covering the significant points from the leading nine chapters as well as the main challenges met during the implementation.