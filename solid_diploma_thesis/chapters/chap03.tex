\chapter{Analysis}
\label{chap:num_3}

This chapter provides a detailed overview of available \solid{} servers, as well as a reasoning for why a server implementation called `node-\solid{}-server` was used as the basis for main thesis project. 

% \TODO{Analysis should not be just about \solid{}. Therefore, everything about \solid{} should be in one subsection}
% \TODO{BTW I suggest creating a new \LaTeX command for spelling \solid{}, because already, it is sometimes \solid, sometimes \solid etc. - consistence is key. What I do is I create a command for each repetitive text, abbreviation etc.}

\section{\solid{} overview}

\subsection{The \solid{} servers}

\solid itself represents a tech stack of complementary standards and data format vocabularies that are currently only available in centralized social media services. It also represents a Specification Document, serving as the main guideline for developers building their own apps and services. However, aside from that \solid{} also refers to a set of servers that implement its specification. 

\subsubsection{gold}

`gold` is a reference Linked Data Platform server for the \solid{} platform. The implementation is written in Go, based on initial work done by William Waites [include reference].


\subsubsection{node-solid-server}

Following solution is an implementation of a server based on \solid{} specifications written entirely in `node.js`.
One of the main advantages is that it could be launched as a \solid{} server on top of the local file-system. Interaction with server can be performed as follows:
\begin{itemize}
	\item Command line tool.
    \item Via `node-solid-server.js` library.
\end{itemize}

The provided server implementation was used as a main \solid{} server for the project due to compliance with main requirements for choosing the server. Those can be described as follows:
\begin{itemize}
	\item \textbf{Easy integration with current LPApps project}, specifically the frontend web project. This allows usage of provided `node-solid-server.js` library.
    \item \textbf{Active maintenance by open-source community}. Aside from that, this server implementation is considered to be a default option suggested by creators of \solid{}.
    \item \textbf{Support of \textit{WebID IdP}, \textit{WebID-TLS}, \textit{WebID-OIDC}}. This implies majority of WebID communications protocols,
    crucial to the user authentication and security aspects of the storage within LPApps.
\end{itemize}

\subsubsection{OpenLink Virtuoso}

Virtuoso Universal Server is a middleware and database engine hybrid that combines the functionality of a traditional Relational database management system (RDBMS), Object-relational database (ORDBMS), virtual database, RDF, XML, free-text, web application server and file server functionality in a single system. Rather than have dedicated servers for each of the aforementioned functionality realms, Virtuoso is a "universal server". It also enables a single multithreaded server process that implements multiple protocols.

The free and open source edition of Virtuoso Universal Server is also known as OpenLink Virtuoso.

\subsection{The \solid{} React development stack}

The main frontend library used in LinkedPipes Applications is React. In order to incorporate easier integration of aspects of \solid{} into the project, React was also used as a main library. Even though, \solid{} community is yet to grow become stable and mature, it already provides a convenient set of libraries for React that were used as a main tools during implementation of the thesis project. 

\subsubsection{solid/react}

The main purpose of this library is to provide the following functionality:

\begin{itemize}
	\item Provide React developers with components to develop fun Solid apps.
    \item Enable React developers to build their own components for Solid.
\end{itemize}

\subsubsection{solid-file-client}

This library provides a simple interface for logging in and out of a Solid data store, maintaining a persistent session, and for managing files and folders. It may be used either directly in the browser or with node/require. The library is based on solid-auth-client and solid-cli, providing an error-handling interface and some convenience shortcuts on top of their methods and providing a common interface to the two modules.

\subsubsection{rdflib}

Javascript RDF library for browsers and Node.js. 
