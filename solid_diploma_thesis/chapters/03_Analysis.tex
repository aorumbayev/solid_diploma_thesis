\chapter{Analysis}
\label{chap:num_3}

%  TODO : start from requirements of the thesis, use overview section that will lead to using solid.

This chapter provides an overview of a so-called concept of centralized data silos. Explains the advantages and disadvantages of centralized data management and control. Afterward, we explain how decentralized privacy oriented data storages could improve the pitfalls of centralized data silos and give full access to their data back to users. The chapter also provides an overview of all major technologies implementing \solid{} specifications, as well as a reasoning for why a server implementation called `node-\solid{}-server` was used as the basis for the main thesis project. 

% \TODO{Analysis should not be just about \solid{}. Therefore, everything about \solid{} should be in one subsection}
% \TODO{BTW I suggest creating a new \LaTeX command for spelling \solid{}, because already, it is sometimes \solid, sometimes \solid etc. - consistence is key. What I do is I create a command for each repetitive text, abbreviation etc.}

\section{SOLID overview}

At the core \solid{} is a set of open specifications. At the current state of the project implementation, the community is mostly concerned about persistence and representation of resources. However, such aspects as identity, authentication and authorization are also vital parts of \solid{}.

A set of these standards are undergoing active implementation stage. Main contributors consisting of the core \solid{} inrupt team, as well as Open Source community, develop the standards into various servers and tools. For instances such as node-solid-server that implements a Solid server in Node, and rdflib.js that allows you to manipulate RDF programmatically.

In addition to this, there is work done on supporting tools, such as Solid React SDK that is created to allow developers to more easily get into developing apps with Solid. As part of this is the style guide that can be reused by others not wanting to use the React parts.

\subsection{The \solid{} servers}

\solid{} itself represents a tech stack of complementary standards and data format vocabularies that are currently only available in centralized social media services. It also represents a Specification Document, serving as the main guideline for developers building their own apps and services. However, aside from that \solid{} also refers to a set of servers that implement its specification. 

\subsubsection{gold}

`gold` is a reference Linked Data Platform server for the \solid{} platform. The implementation is written in Go, based on initial work done by William Waites [include reference].


\subsubsection{node-solid-server}

Following solution is an implementation of a server based on \solid{} specifications written entirely in `node.js`.
One of the main advantages is that it could be launched as a \solid{} server on top of the local file-system. Interaction with server can be performed as follows:
\begin{itemize}
	\item Command line tool.
    \item Via `node-solid-server.js` library.
\end{itemize}

The provided server implementation was used as a main \solid{} server for the project due to compliance with main requirements for choosing the server. Those can be described as follows:
\begin{itemize}
	\item \textbf{Easy integration with current \lpa{} project}, specifically the frontend web project. This allows usage of provided `node-solid-server.js` library.
    \item \textbf{Active maintenance by open-source community}. Aside from that, this server implementation is considered to be a default option suggested by creators of \solid{}.
    \item \textbf{Support of \textit{WebID IdP}, \textit{WebID-TLS}, \textit{WebID-OIDC}}. This implies majority of WebID communications protocols,
    crucial to the user authentication and security aspects of the storage within \lpa{}.
\end{itemize}

\subsubsection{OpenLink Virtuoso}

Virtuoso Universal Server is a middleware and database engine hybrid that combines the functionality of a traditional Relational database management system (RDBMS), Object-relational database (ORDBMS), virtual database, RDF, XML, free-text, web application server and file server functionality in a single system. Rather than have dedicated servers for each of the aforementioned functionality realms, Virtuoso is a "universal server". It also enables a single multithreaded server process that implements multiple protocols.

The free and open source edition of Virtuoso Universal Server is also known as OpenLink Virtuoso.

\subsection{The \solid{} React development stack}

The main frontend library used in LinkedPipes Applications is React. In order to incorporate easier integration of aspects of \solid{} into the project, React was also used as a main library. Even though, \solid{} community is yet to grow become stable and mature, it already provides a convenient set of libraries for React that were used as a main tools during implementation of the thesis project. 

\subsubsection{solid/react}

The main purpose of this library is to provide the following functionality:

\begin{itemize}
	\item Provide React developers with components to develop fun Solid apps.
    \item Enable React developers to build their own components for Solid.
\end{itemize}

\subsubsection{@inrupt/solid-react-components}

This libray is an official SDK provided by Inrupt for developing React Web applications for \solid{}. The package include various dependencies allowing:

\begin{itemize}
	\item Provide React developers with a set of easily customizable components interacting with \solid{} specifications.
    \item Provide a standardized visual design conventions based on Atomic Style.
    \item A set of cli commands for generating a template \solid{} projects.
\end{itemize}

\subsubsection{solid-auth-client}

The main purpose of this low level library is to provide ability to easily perform Authentication operations while interacting with \solid{} pods and servers.

At its core it is a browser library that allows your apps to securely log in to \solid{} data pods and read and write data from them.

\subsubsection{solid-file-client}

This library provides a simple interface for logging in and out of a Solid data store, maintaining a persistent session, and for managing files and folders. It may be used either directly in the browser or with node/require. The library is based on solid-auth-client and solid-cli, providing an error-handling interface and some convenience shortcuts on top of their methods and providing a common interface to the two modules.

\subsubsection{rdflib}

Javascript RDF library for browsers and Node.js. 

\subsubsection{@solid/query-ldflex}

The following library adds support of the LDflex language to \solid{} by:

\begin{itemize}
	\item Providing a JSON-LD context for \solid{}.
    \item Binding a query engine (Comunica).
    \item Exposing useful data paths.
    \item LDflex expressions occur for example on Solid React components, where they make it easy for developers to specify what data they want to show. They can also be used as an expression language in any other Solid project or framework.
\end{itemize}


\subsection{Functional requirements}

The following sections describes the main functional requirements defined for the \lpa{}. 

\subsection{User authentication}

The user of the platform should be able to register an account in the application, log in and log out. Moreover, once logged in, the platform should be provide functionality to create, configure and publish applications visualizing LinkedData. However, to view an application, it is not necessary to be logged in or even registered.

\subsection{Create Application}

The Create Application functionality could be described as a core feature of the LinkedPipes Application.

The platform needs to provide the a tooling to specify the data sources to be utilized so that an instance of an interactive application would be created based on provided data sources. In addition to that, there needs to be four ways to provide the data source:

\begin{itemize}
\item Provide a set of dataset \acrshort{IRI}s which the tool will de-reference to get the dataset.
\item Specify a \acrshort{SPARQL} endpoint from which data will be queried and extracted.
\item Upload a file in \acrshort{TTL} format, containing data source specifications.
\item Use some sample dataset provided by the tool
\end{itemize}

\subsection{Configure Application}

%TODO : rewrite
After the creation of the application, platform also need to provide functionality to configure previously created applications. Each application, depending on the visualization, will have particular settings that can be set. Furthermore, users must be able to control, with the use of filters, which data is to be used and displayed. The available filters should be automatically derived based on the properties and semantics of the data. 

The set of configure operations for filters can be described as follows:

\begin{itemize}
\item Removing whole data filters to ignore the values of specific data properties.
\item Removing selected values of some filters.
\item Setting fixed values for some filters.
\item Set/change a name for the application.
\end{itemize}

\subsection{Publish Application}

Another important requirement is an ability to publish the application and make it publically available for sharing the visualization to anyone via the permanent link. Furthermore, when a third party accesses this \textit{permalink}, the browser should open the LinkedPipes Applications website with the respective application opened, as configured by the publisher.

Besides, when publishing an application I want the tool to offer the possibility to choose one of the below two settings:

\begin{itemize}
\item Use and display cached data, making the published application a fixed view.
\item Regularly refresh the data from the previously chosen data sources of the interactive application.
\end{itemize}

The tool will also need to provide the ability to embed the published view into a data journalist's web page, for example, using an iFrame.

\subsection{Store Application}

Since published applications require to be publically accessible to anyone, the platform needs to have a functionality to persist the configurations of the applications in a secure way. Moreover, the configurations for most of the published applications are represented as LinkedData, therefore the persistent storage tooling needs to be optimized for files in RDF format. 
User needs to be able to browse the stored and published applications and optionally allow other people to collaborate or edit a specific shared application. 

