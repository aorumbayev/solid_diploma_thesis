\chapter*{Introduction}
\addcontentsline{toc}{chapter}{Introduction}


Linked data is a method for publishing structured data in a way that its semantics are also expressed. This semantic description is implemented by the use of vocabularies, which are usually specified by the \acrshort{W3C} as web standards. However, anyone can create and register their vocabulary, for example in an open catalog like \acrshort{LOV}.

Linked data is usually dispersed across many sites on the internet. Each site usually contains only a part of the entire data available, thus a machine or a person trying to interpret the data as a whole needs to link this partial information together using unique entity identifiers shared across the data stores, hence the name `linked' data.

% As more and more . The primary goal of this project is to develop such a tool, in which with a simple web interface any user with no prior experience with concepts of Semantic Web\footnote{\url{https://en.wikipedia.org/wiki/Semantic_Web}} can explore, visualize and interact with Linked Data.

% To be added... 
% Visualizations, linked data, storing data, ways of storing the applications -> decentralization -> solid
 
\section*{Goal of the thesis}
% Solid mentions
Main goal of the following thesis is to provide an efficient decentralized storage for the LinkedPipes Applications project. The key concepts of the project is to provide the most intuitive and convenient experience to lay users to manipulate and visualize the linked data, that is consequently turned into independent applications visualizing that data, that can be easily shared and published. Sharing and publishing visualizers also provide a tooling for organizing people into communities based around certain linked data and consequent applications. That can be considered as a feature for future development of LinkedPipes Applications project. 

\section*{Structure}

In Chapter \ref{chap:num_1}, the overview of the \lpa{} and the main functional requirements are provided. In addition to that, the main terminology is described in general. In Chapter \ref{chap:num_2}, the description of similar existing solutions is given as an introduction to a statement of \lpas{}. In Chapter \ref{chap:num_3}, we review the main development stack of \lpas{} as well as the motivation for choosing specific technologies and tools. In Chapter \ref{chap:num_4}, we start describing the architecture of the \lpas{} storage within the \lpa{}.