\chapter*{Conclusion}
\addcontentsline{toc}{chapter}{Conclusion}

To summarize, in the following work implements a multipart decentralized storage solution for \lpa{} platform based on \solid{} project. The term multipart reffers to implementation of the \lpas{} npm package, the \lpas{} Ontology used to represent \lpa{} visualizer configurations as RDF files and lastly, a set of Storage Components implemented in \lpa{} frontend. 

An overview of related tools in \autoref{chap:num_2} provided a set of software technologies alternative to \solid{}. The chapter also provided a comparison table described in \autoref{sssec:lpa_preliminaries_component_overview} that demonstrated benefits of choosing \solid{} as a core storage technology for \lpa{} platform.

An analysis and description of \lpa{} requirements and \solid{} development toolset in \autoref{chap:num_3} defined the main practical tasks to be achieved by \lpas{} as well as to which \solid{} libraries and frameworks to utilize in implementatioin. The final reasoning expanding the conclusion from \autoref{chap:num_2} on choosing solid was provided at the end of the chapter in \autoref{sssec:why_solid}.

Based on the performed analysis on \solid{} and requirements stated by \lpa{}, \autoref{chap:num_4} described a detailed overview of a designed architecture of \lpas{}. The architecture consisted of three main parts. Firstly, the \autoref{ssec:storage}, provided the design of abstractions for a \lpas{} npm package providing functionality for authenticating (\autoref{sssec:authentication_manager}), operating the resources inside \solid{} PODs (\autoref{sssec:file_manager}) and managing ACL files (\autoref{sssec:access_control_manager_arch}). Afterwards, the \autoref{ssec:lpas_application_ontology_arch} provided an overview of designed OWL Ontology for representing the \lpa{} visualizer configurations as RDF resources inside \solid{} POD as well as describing the benefits of such approach to storing visualizer data. And lastly, \autoref{ssec:lpas_storage_component_design} provided a set of designed user interface mocks complying to stated \lpa{} requirements as well as a detailed overview of the functioinality provided by those user interface components.

Continuing the architecture overview, \autoref{chap:num_5} described the entire implementation of the storage functionality and structured similar to the previous chapter as it implements the designed elements in order with their design and architecture. The \autoref{ssec_storage_package_implementation} provided detailed overview of the \lpas{} package implementation using TypeScript programming language. The \autoref{ssec:storage_ontology_implementation} described the process of implementing and hosting the designed \lpas{} Ontology using Protégé and Ontology open-source tools. The \autoref{sssec:lpas_storage_frontend_implementation} described the implementation of Storage React components inside the \lpa{} frontend codebase. The final renders of components were demonstrated in \autoref{ssec:overview_of_implemented_requirements} by iterating of functional requirements. Lastly, in \autoref{ssec:non_functional_requirements_implementation} an overview of implemented non-functional requirements was provided. Chapter also clearly demonstrated that all defined functional and non-functional requirements of \lpa{} were covered and implemented, thus fulfilling the practical goals of the thesis.

The \autoref{chap:num_6} demonstrated the improvements introduces to \lpa{} after evaluating the platform with fully integrated \lpas{} solution. The chapter also provided the main results and achievements obtained as a part of the implementation stage and evaluation process, such as:
\begin{itemize}
    \item Recognition on official website of the \solid{} project (\autoref{sssec:recogniition_on_solid}).
    \item Comments and short conversations with Sir Tim Berners-Lee in regards to questions on \solid{} specifications asked by author (\autoref{sssec:comments_from_tim}).
    \item Demonstration of user traction over a six month period of continious delivery of the \lpas{} solution. Positive feedback and recognition by \solid{} community members and member of BARTOC organization (\autoref{sssec:user_traction_of_platform}).  
\end{itemize}
 
A detailed overview of the testing of \lpas{} solution was provided in \autoref{chap:num_7}. In \autoref{ssec:lpas_unit_testing} a demonstration of how the \lpas{} package itself was unit tested and automated with Continious Integration and Delivery pipelines using Travis CI. And lastly, in \autoref{ssec:integration_and_delivery} an overview of improvements introduced into \lpa{} automated build pipelines and integration testing of both \lpa{} and \lpas{} solutions was provided.

The whole \lpas{} solution was thoroughly documented both on the npm package side as well as by expanding the \lpa{} documentation to include Storage Components as described in \autoref{chap:num_8}.

\section{Future work}

The \solid{} ecosystem is constantly expanding in its specifications, community, and available technological toolset. Throughout fulfilling the goals of the thesis, many challenges were faced due to the aforementioned constant changes in the \solid{} project. Over time once, \solid{} technology will mature for more advanced production level use cases, and the ecosystem of decentralized social applications will grow. The dependencies such as node-solid-server and solid-auth-client used in \lpas{} package might require significant updates and refactor. Additionally, by the time of finishing the practical part of this thesis, several libraries were introduced by official \solid{} contributors, including a brand new \solid{} server implementation in TypeScript called \texttt{pod-server} \footnote{\url{https://github.com/inrupt/pod-server}} aimed to eventually replace the node-solid-server. Aside from that a library called \texttt{tripledoc} \footnote{\url{https://vincenttunru.gitlab.io/tripledoc/}} was introduced, aiming to potentially become the standard library to interact and manage resources in \solid{} PODs. 

Therefore, potential improvements and future work in \lpas{} package and storage components include the following:
\begin{itemize}
    \item \textit{Gradual refactoring and replacement of node-solid-server}. Replacing the node-solid-server implementation with more stable and actively supported pod-server implementation could introduce more user-friendly experience while performing authentication and manipulation of resources inside \solid{} PODs.
    \item \textit{Gradual integration of tripledoc into \lpas{} package}. As tripledoc potentially offers the same functionality as the \lpas{} package, the potential future work includes replacing certain low-level interactions with rdflib and using tripledoc instead. This can simplify the maintainability of \lpas{} package, simplify unit-testing, and can potentially make it a generic utility for storing and managing any configuration ontologies in \solid{}.
    \item \textit{Improving collaborative editing}. One of the extra features implemented within the scope of \lpas{} was the ability to share published applications within users of \lpa{} platform and let them configure the published application collaboratively. An improved version of that can include real-time editing of the resources more robustly, while the current version discards any simultaneous real-time changes submitted by collaborating users.
    \item \textit{General support of the project}. This, of course, assumes the long term general support of the solution and parts of \lpa{} platform covering the Storage functionality to keep it up to date with all the latest improvements and changes introduces in \solid{} specification.
\end{itemize}

The ambitious goal of decentralizing the World Wide Web set by Sir Tim Berners-Lee is yet to demonstrate its benefits and receive a more comprehensive recognition across non-developer oriented domains and average internet users. However, we believe that the fundamental specifications of \solid{} project that improves upon established Web Standards, an active and passionate community, and a developer-friendly environment and tools to builds decentralized social applications will define the next generation of internet technologies and make it more secure and privacy-oriented.